
\chapter{本期是系列教程的第三期}

这一期的重点是介绍 \LaTeX 语法和命令

\section{写在前面}
\subsection{什么是“够用”}
本系列视频给“够用”设定的边界:在使用现成模版的情况下,知道常用的基本命令和场景,对于高级和进阶的需求,知道去哪里寻找解决方案。

这对于Up(@六边形游乐场)和观众来说都是一种挑战,过于絮叨会看不下去,过于简单又不负责任。我尽量吧,还请坚持过最开始的五分钟。

\subsection{如何进阶}
1. 推荐通过 Overleaf 官方帮助文档学习 \url{https://www.overleaf.com/learn}; 

2. 下载一份 Cheat Sheet 速查手册,在评论区顶部我会提供一份下载链接;

3. 目前这个文档本身(.tex文件和PDF文件),也会在评论区顶部提供下载链接,还请不要忘记一键三连,这次一定!

\section{基础知识}
% 首先,什么是注释
% 右边的界面不会显示这里的内容
% 可以用来给自己写一些备注,或者隐藏掉废弃但是不舍得删除的内容
% 单行注释的语法:以百分号 % 开头

% 下面出现的大部分注释,是给想要进一步进阶的观众朋友提供的信息,视频里都不会去念出来,还请有兴趣的朋友自行下载本文档详细了解

\subsection{命令 Command}
%凡是标题里带着英文翻译的地方,都是搜索关键字,需要进一步了解的同学可以自行去 Google 或 Overleaf 搜索了解更多内容

命令是 \LaTeX 里最基本的指令语句,以反斜杠开头,一般以空格结束,形态是:

\begin{verbatim}
    \command[可选参数]{必选参数}
\end{verbatim}

关于命令的知识:\\
0. 命令区分大小写; \\
1. 有些命令可以没有任何参数,比如说\verb|\newline|;\newline
2. 如果一个命令有可选参数(不论有几个),都应该被写在[和]之间;\newline
3. 如果一个命令有必选参数(不论有几个),都应该被写在\{和\}之间。\newline

% 展示命令的两种方式
% 1. 使用 verbatim 环境,上面已经用到了
% 2. 使用 \verb|| 命令,一般用在行内命令展示,下面会出现

% 分段和换行是不一样的(可以从上面是否缩进看出来)
% 1. 空一行和\par命令是分一段
% 2. \\和\newline命令是换一行

% 以上出现的两次有序列表,都是为了演示的方便
% 当我们真正要使用有序列表的时候,尽量要使用一会下面会介绍的更规范的方式

\subsection{环境 Environment} 
环境指被\verb=\begin=和\verb=\end=命令包围的空间,提供了相对独立的命令作用范围。

环境演示:
\begin{verbatim}
    1. 这里就是 verbatim 环境里面
    2. 环境里的内容会收到这个环境的影响
\end{verbatim}

\subsection{包 Package}
\LaTeX 本体包含了一些命令,但考虑到 \LaTeX 被学术界广泛的使用, 需求多种多样,内置的命令很难满足所有的情况。因此,有了包的概念,每一个包都会提供很多新的命令。

目前 \LaTeX 生态有超过4000种的各种宏包,可以根据需要引入到你的文档里。一会我们下面插入图片的操作中,也会使用到特定的包。

% 除了 Package 以外,还有一种叫做 Class 的文件,可以按需进一步了解

\subsection{代码结构}
一份 \LaTeX 代码(也就是左侧这边的代码),分为两大部分:导言和正文

% 所有文档的完整代码,都应该有这两个部分
%     \documentclass{article}

%     \begin{document}
%       This is a simple example
%     \end{document}

% “导言”区域以 \documentclass{} 开始,以 \begin{document} 结束
% “正文”区域以 \begin{document} 开始,以 \end{document} 结束

“导言” (Preamble) 区域里面可以放一些文档需要引用的外部包; \par
“正文”区域实际上就是一个 document 大环境,绝大部分内容在这里面发生。 \par

\section{常用命令}
\subsection{文档结构 Document Structure}

\verb=\part{}= 篇(最高级) \par
\verb=\chapter{}= 章(第二级) \par
\verb=\section{}= 节(第三级) \par
\verb=\subsection{}= 小节(第四级) \par
\verb=\subsubsection{}= 小小节(第五级) \par

% 还有 \paragraph 和 \subparagraph 不做进一步展开

\subsection{列表 List}
除了控制文档结构的语句外,对我来说,第二常用的就是各种类型的列表。

\vspace{1em} %空一行,只是为了美观

首先,我们展示一下 \textbf{无序列表}:
使用 itemize 环境开启无序列表,使用 \verb|\item| 命令增加列表项。

\begin{itemize}
    \item 这是第一项
    \item 这是第二项
    \item 基本上你可以有任意多个项
    \item 你还可以嵌套,比如说
    \begin{itemize}
        \item 这是一个嵌套的无序列表
        \item 还可以继续嵌套,我们就不演示了
    \end{itemize}
\end{itemize}

\vspace{1em} %空一行,只是为了美观

其次,我们展示一下 \textbf{有序列表}的使用:
使用 enumerate 环境开启无序列表,使用 \verb|\item| 命令增加列表项

\begin{enumerate}
    \item 这是第一项
    \item 这是第二项
    \item 同样可以嵌套,方法和上面类似就不演示了
    \item 有序和无序列表之间也可以相互嵌套
\end{enumerate}

\vspace{1em}

\textbf{知识点}:
\begin{itemize}
    \item 列表里可以嵌套列表,还可以在两种类型列表之前相互嵌套,最多支持四层嵌套
    \item 列表项的序号其实是可以自定义的:使用 \verb|\item[]|代替 \verb|\item|,就可以自定义你喜欢样式,比如说
        \begin{enumerate}
            \item[第一] 这是一项
            \item[第二] 这是一项
        \end{enumerate}
\end{itemize}

% 更多小知识
% 如果想自行定义整个列表对的序号呈现方式,可以在列表环境外面加参数,比如 \begin{itemize}[label=-]
% 如果有自己常用的序号方式,还可以通过 \newlist 命名自定义列表环境

\subsection{图片}
对于插入图片的操作,有两个准备工作:
\begin{itemize}
    \item 首先,需要在文档的导言区域使用命令 \verb|\usepackage{graphicx}| 加载宏包;
    \item 其次,(最好)在导言区域通过命令 \verb|\graphicspath{path}|告诉 \LaTeX  本文档中图片所在文件夹的路径。比如在当前文档中,我们设置了图片路径为 \verb|\graphicspath{{figures/}}|
\end{itemize}

% 添加图片本身支持绝对路径和相对路径,提前声明好图片文件夹、以及把图片放在同一个文件夹里可以避免很多新手错误、降低后期维护和更换图片的难度、降低编译失败后定位问题的难度

\vspace{1em}
\textbf{那么具体怎么插入图片呢?} \par
使用命令 \verb|\includegraphics{文件地址}| \par

效果示意:\newline
\includegraphics[width=\textwidth]{figures/hex.jpg} 

% 可选参数 [width=\textwidth] 的增加,是为了控制图片的宽度
% 可以通过各种参数的控制调整图片的长宽等
% 还可以旋转、裁切之类的

除了像刚才这样直接插入图片,更常见的,考虑到需要对图片进行很多的格式设置,以及我们需要 \LaTeX 帮助给图片增加合适的序号,所以最好创建一个 figure 环境,然后在环境里使用命令插入图片,代码和效果如图 \ref{fig:hex}:\par

\begin{figure}
    \centering
    \includegraphics[width=\textwidth]{figures/hex.jpg} 
    \caption{欢迎关注我的账号}
    \label{fig:hex}
\end{figure}

% 使用 figure 环境可以让 LaTeX 帮我们自动生成图片序号
% \includegraphices 命令中不加.jpg等后缀也可以,LaTeX 会按照不同后缀搜索整个文件夹。所以如果有同名文件,会按照一定规则选取和匹配。
% 我一般就跟直接指定了

关于使用 \LaTeX 内置命令和环境进行绘图,也作为进阶知识,感兴趣的同学可自行搜索学习。

% 另外,实际上有一个重要的概念叫做浮动元素

\newpage

\subsection{表格}
\LaTeX 内置的最基本的表格环境是 tabular,但是非常不好用,所以完全不建议,我们简单演示一下:\par

\begin{tabular}{c|c}
     第一行第一列&第一行第二列  \\
     第二行第一列&第二行第二列 
\end{tabular}

% 表格里面的 \\ 用来在表格里面换行
% 表格头部的必选参数 {c|c}中,两个c代表两列
% 更多细节不再展开,有兴趣可以自行搜索,不过综合来说,不建议使用默认的表格
% 对于新人友好的方式如下:

由于表格,是一个看上去简单、但实际使用过程中事情很多的东西,比如说列宽的设置、合并单元格、各种表格框线的绘制等等,所以使用 \LaTeX 原生的命令绘制好复杂表格的难度过大。\par

因此,最好是使用各种在线 \LaTeX 表格生成器,它们一般都支持你直接上传 Excel 表格,帮你在线转换,也可以在线填写内容生产 \LaTeX 语法的表格。 \par
比如,网页 \url{https://www.tablesgenerator.com/} 提供的工具,如图~\ref{fig:tableTool} 。

\begin{figure}[ht]
    \centering
    \includegraphics[width=\textwidth]{figures/table.png}
    \caption{在线生成表格的工具举例}
    \label{fig:tableTool}
\end{figure}

\newpage

\subsection{数学}
数学公式的 \LaTeX 的最擅长的领域,不仅可以非常方便的撰写出来各种数学公式,最最重要的是可以很好的被自动排版进论文里面,美观大方。\par

数学方面,一般涉及数学符号和单位、数学公式、数学定理三部分内容。
\begin{itemize}
    \item 数学符号和单位涉及国内外或者不同目的的不同标准,最好跟着模版来
    \item 数学公式设计行业公式、行间公式等
    \item 数学定理主要是样式
\end{itemize}
由于以上的这些情况,数学相关的宏包和命令非常多,不过考虑到我们本系列视频的主题“从入门到够用”,这一块我们就不做进一步展开了。而是推荐新人使用在线的可视化工具。\par
一个简单的效果演示:
\begin{align}
  a & = b + c + d + e \\
    & = f + g
\end{align}

% 公式常见的环境包括 equation, align 等
% 定理常见的宏包和环境包括 amsthm, thoerem 等

像表格的处理一样,有需要插入数学公式的同学可以使用一些在线工具,比如,网页 \url{https://www.latexlive.com/} 提供的工具,如图~\ref{fig:mathTool} 。

\begin{figure}[ht]
    \centering
    \includegraphics[width=\textwidth]{figures/math.png}
    \caption{在线生成数学公式的工具举例}
    \label{fig:mathTool}
\end{figure}

还有一些工具,可以直接识别你拍的照片或者手写的公式,比如,网页 \url{https://mathpix.com/} 提供的工具,如图~\ref{fig:handwriteTool} 。

\begin{figure}[ht]
    \centering
    \includegraphics[width=\textwidth]{figures/handwrite.png}
    \caption{在线生成数学公式的工具举例}
    \label{fig:handwriteTool}
\end{figure}

\newpage

\subsection{脚注 Footnote}
这部分就轻松愉快了,是一个我非常喜欢的小功能,但是很方便。可以在文档中的任意位置增加脚注,比如说这里 \footnote{括号里写的内容都会被添加到页面底部,不论这个段落被各种图表挤到了哪一页} \par

具体的命令是 \verb|\footnote{}|

% 当然也可以通过可选参数自定义脚注的编号

\subsection{引用}
引用也非常的简单,在任意需要添加引用的位置,比如这里 \cite{zhangkun1994},使用命令 \verb|\cite{}|(花括号里填写引文信息)\par

当然,前提是你已经把引用的来源,添加到了.bib文件中。这里我们演示一下,在知网或百度学术里面如何获得 BibTeX 的内容,如果需要以后有空我会介绍一下如何在 Zotero 中直接获取这个信息以及如何使用 Zotero 进行文献管理。\footnote{关注@六边形游乐场}

\section{其他}
\subsection{常用符号}
以下是一些可能会用到的常用符号:
\begin{itemize}
    \item  单引号 \verb|`和'|,分别是左单引号`和右单引号'
    \item  双引号 \verb|`和"|,分别是左双引号`和右双引号"
    \item  LaTeX 符号 \verb|\LaTeX|,效果是一个美观的 \LaTeX 符号  
    \item  省略号 \verb|\dots|,效果是一个常见的省略号 \dots
\end{itemize}

\subsection{文字样式}
\begin{itemize}
    \item 加粗 \verb=\textbf{}= \textbf{bold}
    \item 下划线 \verb=\underline{}= \underline{underline}
    \item 斜体 \verb=\textit{}= \textit{italic}
\end{itemize}
