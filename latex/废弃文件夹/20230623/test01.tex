
%文档基本结构
\documentclass{article}

\title{First Tex File}
\author{Andy}
\date{\today}

\usepackage{ctex}
\usepackage{amsmath}
%正文区
\begin{document}
	\maketitle%使得导言区的设置生效
	
	\section{行内公式}
	\subsection{美元符号}
	交换律$a+b=b+a$,如$1+2=2+1$
	\subsection{小括号}
	交换律\(a+b=b+a\),如\(1+2=2+1\)
	\subsection{math环境}
	交换律\begin{math}
		a+b=b+a
	\end{math}
	\section{上下标}
	\subsection{上标}
	$2x^2+3x+5=6$
	\subsection{下标}
	$a_0,a_1,a_{100}$
	\section{希腊字母}
	$\alpha$
	$\beta$
	$\gamma$
	$\epsilon$
	$\pi$
	$\omega$
	
	$\Gamma$
	$\Delta$
	$\Theta$
	$\Pi$
	$\Omega$
	\section{数学函数}
	$\log$
	$\sin$
	$\cos$
	$\arccos$
	$\arcsin$
	$\ln$
	
	$\sin^2x+\cos^2x=1$
	
	$\sqrt{2}$
	$\sqrt{x^2+y^2}$
	$\sqrt{2+\sqrt{2}}$
	$\sqrt[4]{x}$
	\section{分式}
	大约是原体积的$3/4$
	大约是原体积的$\frac{3}{4}$
	\section{行间公式}
	\subsection{$$$$}
	        $$2x^2+5x+3=6$$	
	\subsection{displayment}
	\begin{displaymath}
		2x^2+5x+3=6	
	\end{displaymath}
	\subsection{自动编号公式}
	交换律见式\ref{eq:commutative}
	\begin{equation}
		a+b=b+a \label{eq:commutative}
	\end{equation}
    \subsection{不带自动编号公式}
    \begin{equation*}%需要使用\usepackage{amsmath}
    	a+b=b+a
    \end{equation*}
	
\end{document}